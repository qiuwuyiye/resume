% !TEX TS-program = xelatex
% !TEX encoding = UTF-8 Unicode
% !Mode:: "TeX:UTF-8"

\documentclass[10pt,a4paper]{resume}
\usepackage{geometry}
\usepackage{zh_CN-Adobefonts_external} % Simplified Chinese Support using external fonts (./fonts/zh_CN-Adobe/)

\usepackage{linespacing_fix} % disable extra space before next section
\usepackage[colorlinks,linkcolor=red]{hyperref}
\usepackage{cite}
\geometry{left=0.8cm,right=0.8cm,top=0.6cm,bottom=0.0cm}

\begin{document}
%\bibliography{mycite}


\newcommand{\MyName}[1]{ % Name
		\Huge \usefont{OT1}{phv}{b}{n} \hfill #1 \hfill\
		\par \normalsize \normalfont}
		
\newcommand{\MySlogan}[1]{ % Slogan (optional)
		\large \usefont{OT1}{phv}{m}{n}\hfill \textit{#1} \hfill\
		\par \normalsize \normalfont}
		
\MyName{\textbf{李萍}}
\MySlogan{目标岗位:\textbf{算法工程师}}
% {E-mail}{mobilephone}{homepage}
% be careful of _ in emaill address
\contactInfo{lipingict@gmail.com}{(+86) 130-200-48987}{https://qiuwuyiye.github.io}
% {E-mail}{mobilephone}
% keep the last empty braces!
%\contactInfo{xxx@yuanbin.me}{(+86) 131-221-87xxx}{}
 
\section{\faGraduationCap\  教育背景}
\datedsubsection{\textbf{中国科学院计算技术研究所,计算机,免试保送},北京}{2014 -- 2017}
% \textit{在读硕士}\ 计算机科学与技术
\datedsubsection{\textbf{吉林大学,计算机科学与技术,1\%},长春,吉林}{2010 -- 2014}
% \textit{学士}\ 计算机科学与技术


\vspace{0.5ex}

\section{\faUsers\ 实习经历}
\datedsubsection{\faCheck\ {\textbf{百度凤巢}}\qquad\qquad\qquad\qquad\qquad\qquad 数据挖掘工程师}{2015年5月 -- 2015年8月}
%\begin{itemize}
%  \item 基于凤巢特征抽取系统Adfea和模型训练系统Platform开发了一个Feature评价工具包, 协助了一个产品上线.
%  \item 评估使用时间衰减策略来预测suggestion query 的CPM.
%  \item 调研Kaggle CTR prediction 比赛特征以及方法.
%\end{itemize}
\begin{onehalfspacing}
\textbf{项目概述}\ \ 在FCR\_Model实习期间主要负责模型评估、特征挑选以及特征调研。
\begin{itemize}
  \item {\emph{特征分析工具}}: 基于凤巢特征抽取系统Adfea和模型训练系统Platform开发一个特征评价工具以助力特征挑选。
  \item {\emph{Sug 策略评测}}: 评估时间衰减策略在不同衰减因子以及时间窗口上的Sug CPM预测效果。
 % 负责评估该策略的效果,使用百度半年的sug数据集在不同衰减因子以及时间窗口上做测试,效果略好过线上的baseline.
  \item {\emph{特征调研}}: 调研Kaggle CTR Prediction 比赛中优胜选手使用的特征、连续特征值的处理方式以及使用的模型。
\end{itemize}
\end{onehalfspacing}


\datedsubsection{\faCheck\ {\textbf{微软亚洲工程院}}  \qquad\qquad\qquad\qquad\quad 研发工程师}{2017年4月 -- 2017年5月}
%\begin{itemize}
%  \item 基于凤巢特征抽取系统Adfea和模型训练系统Platform开发了一个Feature评价工具包, 协助了一个产品上线.
%  \item 评估使用时间衰减策略来预测suggestion query 的CPM.
%  \item 调研Kaggle CTR prediction 比赛特征以及方法.
%\end{itemize}
\begin{onehalfspacing}
\textbf{项目概述}\ \ 负责Bing Map中Chain Store数据的采集以及Cortana Answer在若干应用场景下的Bug修复。
\begin{itemize}
  \item {\emph{Chain Store数据采集}}: 构建连锁店基本信息采集工作流,目前可稳定为Bing Map提供连锁店数据。
  \item {\emph{Cortana Answer Bug 修复}}: 修复了Cortana Answer中天气以及连续剧等应用场景下的若干Bug。
 % 负责评估该策略的效果,使用百度半年的sug数据集在不同衰减因子以及时间窗口上做测试,效果略好过线上的baseline.
\end{itemize}
\end{onehalfspacing}

\section{\faUsers\ 项目经历}
\datedsubsection{\faCheck\ {\textbf{图形化大数据机器学习平台 \href{http://bda.space:18080}{BDA Studio}}}}{2015年8月 -- 2016年10月}
%\role{团队项目}{Developer, Spark}
\begin{onehalfspacing}
\textbf{项目概述}\ \ 图形化机器学习平台由可拖拽大数据机器学习平台BDA Studio及大数据机器学习库BDALib 构成。
\begin{itemize}
  \item {\emph{图算法}}: 开发三个图算法(\href{http://bda.space:18080/BDAStudioMonitor.html?job=0000001-160229111630187-oozie-oozi-W}{Pagerank}, \href{http://bda.space:18080/BDAStudioMonitor.html?job=0000015-160229111630187-oozie-oozi-W}{ICmodel}, \href{http://bda.space:18080/BDAStudioMonitor.html?job=0000206-151222123224608-oozie-oozi-W}{KShell}), 相比于graphx原生算法可收敛、可扩展性好、速度快,可支持上10亿规模顶点的图数据挖掘。
  \item {\emph{推荐算法}}: 实现\href{http://bda.space:18080/BDAStudioMonitor.html?job=0000009-160229111630187-oozie-oozi-W}{Factorization Machine} 和\href{http://bda.space:18080/BDAStudioMonitor.html?job=0000021-160229111630187-oozie-oozi-W}{NMF}算法的单机及分布式版本, 在movie-lens上RMSE约0.80。
 % \item 实现\href{http://bda.space:18080/BDAStudioMonitor.html?job=0000021-160229111630187-oozie-oozi-W}{NMF}算法的spark graphx版本,支持模型分布式, 并使用该算法进行过社区发现以及电影评分预测.
 \item {\emph{ETL}}: 完成BDA Studio 的ETL功能,支持Mysql, Hive等异源数据的导入。
\end{itemize}
\end{onehalfspacing}

\datedsubsection{\faCheck\ {\textbf{大规模矩阵分解算法研究与实现}}}{2016年11月 -- 2017年4月}
%\role{团队项目}{Developer, Spark}
\begin{onehalfspacing}
\textbf{毕业论文}\ \ 研究及实现基于图计算框架的大规模矩阵分解算法。
\begin{itemize}
  \item {\emph{在线图划分算法}}: 提出并实现了在线图分割算法BiEdgePartition2D,在保持子图平衡性的同时降低了复制因子。
  \item {\emph{基于模型分布的分布式矩阵分解实现}}: 将待分解矩阵用二部图的形式存储,利用BiEdgePartition2D算法进行分割,基于Pregel图计算框架实现矩阵分解算法,最终减少了约20\%的平均迭代时间。
\end{itemize}
\end{onehalfspacing}

\section{\faUsers\ 比赛经历}
\datedsubsection{\faCheck\ {\textbf{\href{http://bdg.ctyun.cn}{天翼大数据算法应用大赛}}}\qquad\qquad\qquad\qquad\qquad\quad {\textbf{\href{http://bdg.ctyun.cn/algr_detail/PLXBy500006w?pageIndex=al_info}{冠军(1/1111)}}}}{2015年12月 -- 2016年3月}
%\role{团队项目}{Developer, Spark}
\begin{onehalfspacing}
\begin{itemize}
\item {\emph{概述}}: 使用前7周用户每天点击10个视频网站的统计数据,预测用户第八周每天点击视频网站的数量。
\item {\emph{职责}}: 特征调研,特征抽取,特征评价。
\end{itemize}
\end{onehalfspacing}

\datedsubsection{\faCheck\ {\textbf{\href{http://bdg.ctyun.cn}{Google Girl Hackthon 2017}}} \qquad\qquad\qquad\qquad\qquad 最有影响力奖}{2017年3月}
%\role{团队项目}{Developer, Spark}
\begin{onehalfspacing}
\begin{itemize}
\item {\emph{概述}}: 在三天时间内独自设计并实现了基于表达学习的人才检索系统Talent Search。
  \item {\emph{模型训练}}:基于表达学习,将用户的评论信息和人才需求映射到相同的空间,进行人才匹配。
\end{itemize}
\end{onehalfspacing}
%\datedsubsection{\textbf{新浪微博相似用户关系展示}}{2015 年5月 -- 至今}
%\role{Developer, Python, Django, Networkx}{个人项目}
%\begin{onehalfspacing}
%微型网站,查找某一user的相似新浪用户,并展示出之间的链接关系。https://github.com/qiuwuyiye/simrank
%\begin{itemize}
%\item 从hive表中读取相似关系,抽取核心用户,并保存它们之间的图结构。
%\item 根据ID抽取用户的username.
%\item 支撑相似用户follow 关系图形化显示.
%\end{itemize}
%\end{onehalfspacing}


% Reference Test
%\datedsubsection{\textbf{Paper Title\cite{zaharia2012resilient}}}{May. 2015}
%An xxx optimized for xxx\cite{verma2015large}
%\begin{itemize}
%  \item main contribution
%\end{itemize}

% increase linespacing [parsep=0.5ex]



\section{\faHeartO\ 论文及获奖情况}

\datedline{\textit{分布式算法实现比较研究:数据分布与模型分布}, CCIR2016}{2016年6月}
\datedline{\textit{Ease the Process of Machine Learning with Dataflow}, CIKM2016}{2016年5月}
\datedline{\textit{Sesssion Segmentation Method Based on Naive Bayes}, ISTP检索}{2012 年9 月}
\datedline{\textit{Sesssion Segmentation Method Based on COBWEB}, EI 检索}{2012 年6 月}
%\datedline{\textit{{\textbf{\href{http://bdg.ctyun.cn}{天翼大数据算法应用大赛}}}}, 冠军(1/1111)}{2012 年6 月}


\datedline{\textit{连续三年国家奖学金}, 吉林大学}{2011年 -- 2013年}

%\section{\faCogs\ 技能}
%\begin{itemize}[parsep=0.5ex]
%  \item 编程语言: Python =  C++ > Scala > Shell
  %\item 语言: 英语 - 熟练(六级) , 日语(JLPT N2)
%\end{itemize}
%\section{\faTag\ 自我评价}
%\ \ \ \ \\踏实认真,有团队协作的能力和沟通协调能力。有扎实的编程功底以及较丰富的大数据处理经验。熟悉大规模数据挖掘和机器学习,能熟练使用Hadoop, Spark等大数据平台。

\end{document}
