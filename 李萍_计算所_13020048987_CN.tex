% !TEX TS-program = xelatex
% !TEX encoding = UTF-8 Unicode
% !Mode:: "TeX:UTF-8"

\documentclass[13pt,a4paper]{resume}
\usepackage{geometry}
\usepackage{zh_CN-Adobefonts_external} % Simplified Chinese Support using external fonts (./fonts/zh_CN-Adobe/)
%\usepackage{zh_CN-Adobefonts_internal} % Simplified Chinese Support using system fonts
\usepackage{linespacing_fix} % disable extra space before next section
\usepackage[colorlinks,linkcolor=red]{hyperref}
\usepackage{cite}
\geometry{left=1.0cm,right=1.0cm,top=1.0cm,bottom=1.0cm}
\begin{document}
\pagenumbering{gobble} % suppress displaying page number

\name{李萍}

% {E-mail}{mobilephone}{homepage}
% be careful of _ in emaill address
\contactInfo{lipingict@gmail.com}{(+86) 130-200-48987}{http://www.lipingict.com}
% {E-mail}{mobilephone}
% keep the last empty braces!
%\contactInfo{xxx@yuanbin.me}{(+86) 131-221-87xxx}{}
 
\section{\faGraduationCap\  教育背景}
\datedsubsection{\textbf{中国科学院计算技术研究所,  免试保送}, 北京}{2014 -- 2017}
\textit{在读硕士}\ 计算机科学与技术
\datedsubsection{\textbf{吉林大学, 1\%}, 长春, 吉林}{2010 -- 2014}
\textit{学士}\ 计算机科学与技术

%\section{\faTag\ 自我评价}
%\ \ \ \ \\踏实认真,有团队协作的能力和沟通协调的能力。有扎实的编程功底以及较丰富的大数据处理经验。熟悉Hadoop, Spark等大数据平台,能熟练使用MapReduce,BSP分布式计算框架。对机器学习等算法有一定认识。


% increase linespacing [parsep=0.5ex]
\section{\faCogs\ 技能}
\begin{itemize}[parsep=0.5ex]
  \item 编程语言: Python == C++ > Scala > Shell
  \item 平台: Spark, Hadoop
  \item 语言: 英语 - 熟练(六级) , 日语(JLPT N2)

 % \item 统计: numpy, scikit-learn, pandas, matplotlib
\end{itemize}

\section{\faUsers\ 实习/项目经历}
\datedsubsection{\faCheck\ {\textbf{百度凤巢}}  }{2015年5月 -- 2015年8月}
\textbf{个人职责}\ \ 数据挖掘工程师
\newline
%\begin{itemize}
%  \item 基于凤巢特征抽取系统Adfea和模型训练系统Platform开发了一个Feature评价工具包, 协助了一个产品上线.
%  \item 评估使用时间衰减策略来预测suggestion query 的CPM.
%  \item 调研Kaggle CTR prediction 比赛特征以及方法.
%\end{itemize}
\begin{onehalfspacing}
\textbf{项目概述}\ \ 在FCR\_Model实习期间主要负责模型评估、特征挑选以及特征调研.
\begin{itemize}
  \item {\emph{特征分析工具}}: 凤巢没有统一的特征分析工具,特征的筛选是自由组合,筛选效率较低,基于凤巢特征抽取系统Adfea和模型训练系统Platform开发一个Feature评价工具包用来分析特征从而助力特征挑选.
  \item {\emph{sug 模型评测}}: 线上使用的sug cpm预估模型效率较低,使用时间衰减策略来预测suggestion query 的CPM的可以极大地提高效率. 负责评估该策略的效果,使用百度半年的sug数据集在不同衰减因子以及时间窗口上做测试,效果略好过线上的baseline.
  \item {\emph{特征调研}}: 调研Kaggle CTR prediction 比赛中优胜选手使用的特征, 连续特征值的处理以及使用的模型,并写出详细的调查报告.
\end{itemize}
\end{onehalfspacing}

\datedsubsection{\faCheck\ {\textbf{图形化大数据机器学习平台 \href{http://bda.space:18080}{BDA Studio}}}}{2015年8月 -- 至今}
%\role{团队项目}{Developer, Spark}
\textbf{个人职责}\ \ 算法工程师\newline
\begin{onehalfspacing}
\textbf{项目概述}\ \ 图形化机器学习平台由BDA Studio以及BDALib 构成,分别是大数据机器学习库和可拖拽大数据机器学习平台BDA Studio.
\begin{itemize}
  \item {\emph{图算法}}: 开发三个图算法(\href{http://bda.space:18080/BDAStudioMonitor.html?job=0000001-160229111630187-oozie-oozi-W}{Pagerank}, \href{http://bda.space:18080/BDAStudioMonitor.html?job=0000015-160229111630187-oozie-oozi-W}{ICmodel}, \href{http://bda.space:18080/BDAStudioMonitor.html?job=0000206-151222123224608-oozie-oozi-W}{KShell}), 相比于graphx原生算法相比可收敛,具有可扩展性、速度快等特点,可以支持上10亿规模顶点的图数据挖掘.
  \item {\emph{推荐算法}}: 实现\href{http://bda.space:18080/BDAStudioMonitor.html?job=0000009-160229111630187-oozie-oozi-W}{Factorization Machine} 和\href{http://bda.space:18080/BDAStudioMonitor.html?job=0000021-160229111630187-oozie-oozi-W}{NMF}算法的local 版本、spark shared 版本、spark graphx版本, 并使用该算法对movie-lens数据集进行评分预测.
 % \item 实现\href{http://bda.space:18080/BDAStudioMonitor.html?job=0000021-160229111630187-oozie-oozi-W}{NMF}算法的spark graphx版本,支持模型分布式, 并使用该算法进行过社区发现以及电影评分预测.
 \item {\emph{ETL}}: 完成BDA Studio 的ETL功能,支持Mysql, Hive等异源数据的导入.
\end{itemize}
\end{onehalfspacing}

\datedsubsection{\faCheck\ {\textbf{\href{http://bdg.ctyun.cn}{天翼大数据算法用用大赛}}}}{2015年12月 -- 2016.3}
%\role{团队项目}{Developer, Spark}
\begin{onehalfspacing}
\begin{itemize}
\item {\emph{概述}}: 使用前7周用户每天点击10个视频网站的统计数据,预测用户第八周每天点击视频网站的数量。
\item {\emph{职责}}: 特征调研,特征抽取,特征评价
  \item {\emph{成绩}}: 比赛历时两个月,第一赛季排行榜第九名, 第二赛季在{\textbf{1111名}}队伍中{\textbf{\href{http://bdg.ctyun.cn/algr_detail/PLXBy500006w?pageIndex=al_info}{斩获第一}}}。
  \item {\emph{技术}}: Spark, Xgboost, GBDT
\end{itemize}
\end{onehalfspacing}

%\datedsubsection{\textbf{新浪微博相似用户关系展示}}{2015 年5月 -- 至今}
%\role{Developer, Python, Django, Networkx}{个人项目}
%\begin{onehalfspacing}
%微型网站,查找某一user的相似新浪用户,并展示出之间的链接关系。https://github.com/qiuwuyiye/simrank
%\begin{itemize}
%\item 从hive表中读取相似关系,抽取核心用户,并保存它们之间的图结构。
%\item 根据ID抽取用户的username.
%\item 支撑相似用户follow 关系图形化显示.
%\end{itemize}
%\end{onehalfspacing}


% Reference Test
%\datedsubsection{\textbf{Paper Title\cite{zaharia2012resilient}}}{May. 2015}
%An xxx optimized for xxx\cite{verma2015large}
%\begin{itemize}
%  \item main contribution
%\end{itemize}


\section{\faHeartO\ 论文及获奖情况}
\datedline{\textit{Sesssion Segmentation Method Based on COBWEB}, EI 检索}{2012 年6 月}
\datedline{\textit{Sesssion Segmentation Method Based on Naive Bayes}, ISTP检索}{2012 年9 月}
\datedline{\textit{连续四年国家奖学金}, 吉林大学}{2011年 -- 2013年}
\datedline{\textit{连续四年吉林大学优秀学生}, 吉林大学}{2011年 -- 2013年}
%
%\section{\faInfo\ 其他}
%% increase linespacing [parsep=0.5ex]
%\begin{itemize}[parsep=0.5ex]
%  \item 技术博客: http://www.lipingict.com
%  % \item GitHub: https://github.com/qiuwuyiye
%  \item 语言: 英语 - 熟练(六级 555)
%  \item 语言: 日语(JLPT N2)
%  \item 爱好:乒乓球, 游泳,摄影
%\end{itemize}

%% Reference
%\newpage
%\bibliographystyle{IEEETran}
%\bibliography{mycite}
\end{document}
